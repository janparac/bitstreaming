\documentclass[12pt,a4paper,openright,twoside]{article}

\usepackage[italian]{babel}
%\usepackage[latin1]{inputenc}
\usepackage[utf8x]{inputenc}
%\usepackage[applemac]{inputenc}

\usepackage[]{hyperref} %collegamenti ipertestuali
\usepackage{xcolor} 
\hypersetup{ %personalizzazione collegamenti ipertestuali
    colorlinks = false,
    linkbordercolor = {yellow},
}

\usepackage[centertags]{amsmath}
\numberwithin{equation}{section} %numerare le formula fino alla sezione (scrivere SUBSECTION per arrivare alla sottosezione)
\usepackage{amsfonts}
\usepackage{amssymb}
\usepackage{amsthm}
\usepackage{amstext}
\usepackage{enumitem}%per gli elenchi numerati romani
\usepackage{multirow}%per tabelle con righe unite
\usepackage{colortbl}
\usepackage[version=2]{mhchem} %nuclidi
\usepackage{footnote} %per fare note in tabella
\makesavenoteenv{table} %per fare note in tabella
\usepackage[T1]{fontenc} % mette le due virgolette uguali e dritte (forse)

\usepackage[pdftex]{graphicx}%per le immagini

\DeclareGraphicsExtensions{.pdf,.png,.jpg,.mps}
\usepackage{fancyhdr}
\usepackage{indentfirst}
\usepackage{newlfont}
\usepackage[small,bf]{caption}
\usepackage{lscape}
\usepackage{multicol}
\usepackage{units}%per il segno di frazione figo

\usepackage{subfigure}%matrici di immagini
\usepackage{wrapfig}%per le immaggine nel a lato testo
\usepackage[percent]{overpic}%per mettere il testo davanti alle figure

%PAGE formatting
\setlength{\hoffset}{-25,4mm}
\setlength{\captionmargin}{10pt}
\evensidemargin=17mm
\oddsidemargin=17mm
\linespread{1}
\textwidth=175mm
\textheight=230mm

\hyphenation{}


\pagestyle{fancy}
%\renewcommand{\chaptermark}[1]{\markboth{\thechapter.\ #1}{}}
%\renewcommand{\sectionmark}[1]{\markright{\thesection \ #1}{}}
\fancyhf{}
\fancypagestyle{plain}{\fancyhead{}}
\fancyhead[LE, RO] {\bfseries\thepage}
\fancyhead[LO]{\bfseries\rightmark}
\fancyhead[RE]{\bfseries\leftmark}
\renewcommand{\headrulewidth}{0.5pt}
\renewcommand{\footrulewidth}{0pt}
\addto\captionsitalian{\renewcommand{\chaptername}{}}

\title{\textbf{ Taratura dell'MCA per Spettroscopia-$\gamma$ con rivelatore \textit{NaI(Tl)}}}
\date{\small{ Turno di misura\\
	22/2/2017}}
\author{Gruppo 3: \\  
		Stefano Paracchino 
		}



\begin{document}
\maketitle
    

%--------------------INDICE---------------------------------------------------------------------
\tableofcontents                        %crea l'indice
%%%%%%%%%%%%%%%%%%%%%%%%%%%%%%%%%%%%%%%%%imposta l'intestazione di pagina

%%%%%%%%%%%%%%%%%%%%%%%%%%%%%%%%%%%%%%%%%non numera l'ultima pagina sinistra





%--------------------------------------------------------------------------------------------------------
\pagebreak

\pagenumbering{arabic}      

\section{Scopo dell'esperienza}
L'obiettivo dell'esperienza è caratterizzare il sistema di acquisizione per spettroscopia-$\gamma$ tramite la taratura dell' \textit{MCA} e lo studio della risoluzione del rivelatore \textit{NaI(Tl)}. 
Con la taratura ottenuta si potrà quindi analizzare lo spettro di energia prodotto da un isotopo radioattivo.
Da quest'ultimo si procederà al riconoscimento di una sorgente radioattiva ignota e alla misura della massa dell'elettrone a diverse velocità che conduce alla scoperta sperimentale della \textit{Relatività Speciale}.

\section{Richiami teorici}
\subsection{Sorgenti radioattive}
Il decadimento-$\gamma$ si verifica in seguito alla trasmutazione di un nucleo da uno stato eccitato ad uno stato di energia inferiore senza cambiamento della specie chimica.
La conservazione dell'energia prevede l'emissione di un fotone di energia pari alla differenza di energia dei due stati quantistici. Le energie considerate sono quindi quantizzate e il raggio-$\gamma$ ad alta energia (in questa esperienza $\approx 100 keV \div 1.5 MeV $) che viene emesso, sarà l' "impronta digitale" del corrispondente isotopo radioattivo.

Di solito gli stati eccitati che generano decadimenti-$\gamma$, derivano a loro volta in cascata da altri decadimenti ($\alpha, \beta , EC ,...$). Inoltre uno stato eccitato può passare al ground-state attraverso uno o più stati metastabili generando contemporaneamente ($\approx 10^{-1} ps$) più raggi $\gamma$ caratteristici. Infine va ricordato che quasi sempre gli isotopi presentano diversi decadimenti in competizione fra loro (detti multimodali); nei casi trattati in questa esperienza l'isotopo ha sempre un modo di decadimento nettamente più probabile ($>90 \% $) e quindi gli altri modi verranno ignorati dato lo scopo prefissato.
I decadimenti radioattivi avvengono con un certa frequenza temporale detta \textbf{attività} tipica della sorgente e variabile nel tempo. Non è rilevante ai fini di questa esperienza misurare le attività delle sorgenti, basta tenere conto del fatto che sono dell'ordine del $KBq$.
I dati rilevanti delle sorgenti disponibili in laboratorio  sono riassunti in Tabella (\ref{nuclidi}) e saranno i dati noti con cui calibrare l'MCA.

\begin{table}[]

\renewcommand\arraystretch{1.2}
\centering 
\begin{tabular}{|c|c|c|c|c|c|}
\hline
Nome & $t_{\frac{1}{2}}$ [y] & Decadimenti  & Progenie & Picchi principali [MeV]\footnote{i valori sono assunti senza errore}& C-Edge \footnote{Sono qui riportate solo le spalle Compton (Compton Edge) "visibili", come spiegato in App.\ref{a}} [MeV]\\ \hline
      $\ce{^{60}_{27} \mathbf{Co}}$  & 5.62   &    $\beta^{-} \rightarrow \gamma_1 \rightarrow \gamma_2 $         &     $\ce{^{60}_{28}Ni}$     &  1.173, 1.332  & 0.963 ($\gamma =$ 1.173) \\ \hline
      
       $\ce{^{22}_{11}\mathbf{Na}}$    &   2.62       &    $\beta^{+} \rightarrow \gamma    $    &    $ \ce{^{22}_{10}Ne}$     &         0.511 \footnote{Questo fotone non è un decadimento-$\gamma$ del $\ce{^{22}Na}$ bensì deriva dall'annichilazione del positrone del decadimento $\beta^{+} $ con un elettrone della sorgente stessa. }   , 1.275    & 0.339, 1.061 \\ \hline
       $\ce{^{137}_{55}\mathbf{Cs}}$   &     30    &    $\beta^{-} \rightarrow \gamma    $         &    $\ce{^{137}_{56}Ba}$      &     0.662        &  0.477    \\ \hline
       $\ce{^{133}_{56}\mathbf{Ba}}$    &     10.5    &       $EC \rightarrow \gamma_1 ...\rightarrow \gamma_n   $   &    $\ce{^{133}_{55}Cs}$      &         0.081, 0.297, 0.356, 0.382    &  - \\ \hline
       
\end{tabular}
     \caption{Nuclidi impiegati nell'esperienza}
     \label{nuclidi}
\end{table}


\subsection{Ricavare la Relatività Speciale dallo Scattering Compton}
Se oltre alle leggi di conservazione di energia e quantità di moto si assume come noto lo scattering Compton "classico" (il quale richiede a sua volta si accettare la relazione massa-radiazione $E=h \nu$) si può giungere alla \textit{scoperta sperimentale} del concetto di massa relativistica, senza dover sviluppare la teoria della Relatività dai principi primi.

Come viene spiegato nel paragrafo \ref{sper}.\ref{Compt} con la strumentazione di questo esperimento è possibile misurare l'energia della \textit{spalla Compton} e quindi l'energia cinetica $T_e$ che assume l'elettrone nel caso di back-scattering con il fotone tramite la formula (\ref{Te}).
Ci si aspetta usando la meccanica classica di poter quindi calcolare la massa dell'elettrone invertendo la formula dell'energia cinetica $T_e=\frac{p_{e} ^2}{2 m_e}$. La quantità di moto si ricava imponendo le leggi di conservazione allo scattering Compton ottenendo $p_e c=2E_{\gamma}-T_e$.
Mettendo la seconda nella prima si ottiene l'espressione per $m_e$:

\begin{equation} \label{massa}
m_e c^2=\frac{p_{e} ^2 c^2}{2 T_e} = \frac{(2E_{ \gamma }-T_e)^2}{2T_e}
\end{equation}

L'ultimo termine non va visto come una funzione, ma appunto come una "espressione sperimentale". Nel paragrafo \ref{rel} si graficano i valori di $m_e c^2$ ottenuti introducendo nella \ref{massa} le misure sperimentali di $E_{\gamma}$ e dei corrispondenti $T_e$.

La meccanica classica prevede un andamento costante di $m_e c^2$ al variare di $T_e$ ; invece apparirà (forse!) un andamento lineare con intercetta di $m_0 c^2 =0.511 MeV$ e pendenza $k=1/2$. Questi saranno i coefficienti che si ottengono \underline{sperimentalmente} dal Fit lineare (con relativa incertezza). Se si sostituisce l'espressione della retta nel membro a sinistra della (\ref{massa}) si ottiene $p^2 c^2 = 2 T_{e} m_0 c^2 + T_{e} ^2 $ e completando il quadrato:

\begin{equation}
p^2 c^2 + (m_0 c^2 )^2 = (T + m_0 c^2)^2 = E^2
\end{equation}
 che è la nota relazione relativista fra energia e momento.
\section{Apparato sperimentale} \label{sper}

\begin{figure}[hbtp]
\centering
\includegraphics[scale=0.5,angle=-90]{immagini/Strum_FLAT.pdf}
\caption{Apparato Strumentale}
\label{strum}
\end{figure}


In riferimento alla figura \ref{strum} si possono vedere le seguenti componenti della strumentazione:
\begin{enumerate}
\item \underline{Rivelatore NaI(Tl) :} è uno scintillatore inorganico molto diffuso nella spettroscopia-$\gamma$ grazie alla elevata efficienza luminosa (risoluzione di picchi deboli) e alla elevata probabilità di effetto fotoelettrico (fenomeno necessario per la formazione dello spettro caratteristico) dovuta all'alto numero atomico Z ($\sigma_{p.e.} \approx Z^5$)

 Si dispone di un cilindro 3"x 3"  contenuto in guscio protettivo di Al. La sorgente radioattiva può essere posta a pochi centimetri di fronte al rivelatore su apposito portacampione. Il tutto è contenuto dentro uno schermo al Piombo.


 Il gamma incidente scambia energia con un elettrone del cristallo. Grazie alla particolare struttura a bande e alle trappole introdotte dal drogante, l'energia dell' elettrone scatterato verrà tradotta in un numero proporzionale di fotoni $N_{ph}$ (efficienza luminosa $QE \approx 1 \frac{ph}{100eV} = 4 \cdot 10^4 \frac{ph}{MeV}  \quad @ 300 K$ , $\lambda_ {max} = 410 nm $) emessi con tipica risposta esponenziale decrescente, la cui costante decadimento è $\tau_{s} \approx 0.2 \mu s$. In prima approssimazione il rate $L(t)$ di fotoni emessi sarà:
\begin{equation}
L(t)=\frac{N_{ph}}{\tau_{s}}e^{- \frac{t}{\tau_{s}}}
\end{equation}
Il cristallo è trasparente ai fotoni (di bassa energia) che esso stesso genera; il sottile rivestimento Alluminio, oltre a proteggere il NaI(Tl), è trasparente ai Gamma e riflettente ai fotoni visibili.h
In base a quanto detto sopra, lo scintillatore non "vede" i raggi-$\gamma$ bensì "vede" solo gli elettroni scatterati dall'urto con essi.
Questo punto è fondamentale per interpretare i segnali prodotti dallo scintillatore, ovvero a sua volta il grafico prodotto dall' MCA.

 I processi di interazione radiazione materia di interesse in questa esperienza sono:
\begin{enumerate}
\item \underline{Effetto Fotoelettrico (PE):} il fotone ($h \nu \approx MeV$) cede totalmente la sua energia ad un elettrone legato della shell K o L  ($L_e \approx keV$). Quest'ultimo viene emesso con energia $E_e=h \nu - L_e \approx h \nu$. Ci si aspetta quindi un andamento deltiforme detto \textbf{fotopicco} del grafico centrato nel valore $E_e$. Ciò che si osserva è un aspetto di tipo gaussiano. Questo è dovuto alla fluttuazione statistica del numero $N_{ph}$ corrispondenti ad un determinato elettrone scatterato (oltre che alle fluttuazioni delle risposte della catena elettronica seguente). Il rapporto fra la FWMH e l'energia del fotopicco si chiama \textbf{risoluzione} del rivelatore ed è indice delle prestazione dello strumento e della capacità di risolvere picchi vicini fra loro. Se si pensa a un valore di picco come allo "scintillatore che conta fotoni" allora seguirà una distribuzione poissoniana e la risoluzione si può riscrivere come $\propto \frac{\sqrt{N}}{N} \propto \frac{1}{\sqrt{N}} \propto \frac{1}{\sqrt{E}} $ quindi avrà un andamento lentamente decrescente al crescere dell'energia.

\item \label{Compt} \underline{Effetto Compton :} urto elastico relativistico tra fotone ed elettrone "libero", assunto a riposo. L'energia $T_e$ dell'elettrone dopo l'urto dipende dall'angolo $\theta$ di scattering del fotone. Si ha:
\begin{equation}\label{back}
T_e=h \nu - h \nu ' = h \nu - \frac{h \nu }{1+(1-cos \theta )\frac{h \nu}{m_e c^2}}
\end{equation}
Qui a differenza dell'effetto PE si ha un range continuo di energie "tradotte in scintille" da un valore minimo $T_{e} ^{min}$ ad un valore massimo  $T_{e} ^{max}$ al variare dell'angolo $\theta $:
\begin{gather}
\theta=0° \qquad \text{fotone in avanti}\qquad \rightarrow \qquad T_{e} ^{min}=0\\
\theta=180° \qquad \textbf{backscattering} \qquad \rightarrow \qquad T_{e} ^{max}=h \nu \left(1- \frac{1}{1+2 \frac{h \nu}{m_e c^2}} \right) 
\label{Te}
\end{gather}
Se in prima approssimazione si assume che lo scattering Compton sia equiprobabile per ogni $\theta$ ci si aspetta una distribuzione di eventi con un andamento simil rettangolare che parte da $E=0$ e termina in $E=T_{e} ^{max}$ formando uno spigolo detto comunemente \textbf{spalla Compton}.
\item \underline{Produzione di Coppie :} interazione fotone-nucleo che porta alla creazione di una coppia elettrone-positrone e quindi di un "gamma interno" da $0,511 MeV$ in seguito all'annichilazione. Questo processo non è osservato nella presente esperienza come si deduce da Fig.\ref{sezione}.

\end{enumerate}

I processi (a), (b), (c) sono fra loro concorrenti, la rispettiva sezione d'urto è fortemente dipendente dall'energia del raggio-$\gamma$ incidente e è riportata in  Fig. \ref{sezione}.

\begin{figure}[hbtp]
\centering
\includegraphics[scale=0.5]{immagini/section.jpg}
\caption{Sezione d'urto raggi$\gamma$-$e^-$ per NaI(Tl)}
\label{sezione}
\end{figure}

L'intervallo da tenere in considerazione per la seguente esperienza è $10^5 \div 10^6 keV$. Qui si nota che la produzione di coppie è praticamente nulla. L'effetto Compton ha in generale maggiore probabilità dell'effetto PE. Tuttavia si osserva sperimentalmente un fotopicco molto più alto del \textit{rettangolo Compton}, questo è dovuto al fatto che il fotone uscente da un urto compton può di nuovo fare nell'istante successivo un nuovo urto Compton : i fotoni visibili verranno "letti contemporaneamente" e "confusi" con un evento PE.
Un fenomeno analogo accade anche per le varie combinazioni di picchi (e.g. il doppietto del $\ce{^{60}_{}Co}$) dando origine al cosiddetto \textbf{picco-somma}.

\item \label{PMT}\underline{Tubo fotomoltiplicatore (PMT) :} il tubo Silena mod.206 è collegato allo scintillatore in modo tale da tradurre i fotoni emessi in segnale elettrico. Questo è realizzato con la catena di dinodi, alimentati con il partitore dell'alta tensione $HV$ (cfr. Par \ref{HV}). Il catodo di questa catena è posto immediatamente dopo la finestra ottica del NaI(Tl)ed è detto fotocatodo. Esso ha un lavoro di estrazione opportuno affinchè il fotone visibile produca l'emissione di un elettrone (efficienza $E_{PC} \approx 25 \% $). Quest'ultimo entrando nella catena di dinodi raggiungerà un'amplificazione finale di $K_{PM} = 10^6 \div 10^8 $. Se i vari stadi non distorcono il segnale luminoso dello scintillatore, la corrente generata sarà 
\begin{equation}
I(t)=\frac{Q_{PMT}}{\tau_{s}}e^{- \frac{t}{\tau_{s}}}
\end{equation}
dove $Q_{PMT}=N_{ph}\cdot f \cdot E_{PC} \cdot K_{PM} \cdot q$ dove $q$ è la carica elettronica e $f$ la frazione dei fotoni prodotti che effettivamente giungono sul fotocatodo in seguito a riflessioni totali (tipicamente $f \approx 0.2$).

 Ad esempio per un gamma da 1MeV si ha:
 $Q_{PMT} \approx 4 \cdot 10^4 \cdot 0.2 \cdot 0.25 \cdot 10^6 \cdot 1.6 \cdot 10^{-19} \approx 10^{-10} \; [C]$

Il segnale verrà letto sulla caduta di tensione circuito RC-parallelo dell'anodo. Se si suppone una costante di tempo $RC=\theta << \tau_{s}$ (condizione di derivazione) si dimostra che il segnale avrà forma:
\begin{equation}
V(t)=\frac{Q_{PMT}}{C} \frac{\theta}{\tau_{s}} e^{- \frac{t}{\tau_{s}}}
\end{equation}
Con $R=50 \Omega$, $C=20 pF$ ci si aspetta un segnale dell'ordine del $(10^{-10} \cdot 50) /(2 \cdot 10^{-7}) \approx $\textbf{mV} di ampiezza e dell'ordine del \textbf{ us} di durata.

\item \underline{Alimentatore:} modulo NHQ-203M fornisce l'alta tensione al PMT.
\item \underline{Amplificatore:} installato sul Rack,si trova il modulo di amplificazione ORTEC-485, il quale inverte il segnale in ingresso e lo amplifica di un fattore regolabile tramite le due manopole  \textit{coarse gain} e \textit{fine gain}. 

\item \underline{Contatore:} nel medesimo crate dell'amplificatore si trova il contatore ORTEC SCALER-484, il quale conta gli impulsi in ingresso che superano una determinata \textit{threshold} settata da manopola, nella finestra temporale impostata. Il valore di conteggio sarà ovviamente proporzionale all'attività della sorgente.

\item \underline{MCA:} il Multi Channel Analyzer (modello Pocket MCA della Amptek) è l'elettronica fondamentale per produrre un grafico di \textit{conteggi vs energia}. Esso è un ADC che divide un range di tensione $V$ in $(2^n)$ canali di ampiezza $\frac{V}{2^n}$ (con valori discreti da $0$ a $\left( V-\frac{V}{2^n} \right)$), con $n$ scelto dall'operatore.  Il modulo che accetta in ingresso il cavo BNC del segnale (amplificato) con tensione massima di $10V$ e in uscita è collegato al PC con software di controllo a cui trasmette il valore binario ottenuto (questa è anche detta modalità PHA, Pulse Height Analyzer). 

\item \underline{Computer:} il modulo MCA in uscita presenta il cavo VGA con il quale "dice" al PC in quale canale "cade" 
 il segnale appena rilevato. Il software di controllo della Amptek installato incrementa i contatori dei singoli canali e quindi permette di visualizzare on-line la costruzione online del grafico \textit{conteggi vs energia}.

Dispone inoltre di funzioni integrate per estrapolare informazioni dal grafico: si può selezionare un intervallo delle ascisse detto "Range of Interest" (ROI), nel quale il software calcola con formule statistiche interne la posizione del picco, FWMH, conteggi totali, rate,...
I dati possono venire esportati per essere elaborati

\item \underline{Oscilloscopio:} Oscilloscopio analogico Tektronix $BP 100 MHz$ utilizzato per il controllo del segnale in uscita dal PMT. La lettura del segnale è effettuata con opportuno tappo a T da $50 \Omega$ per accoppiare l'impedenza caratteristica del cavo coassiale impiegato che è comunemente di $50 \Omega$ a sua volta.

\end{enumerate}
\section{Procedura sperimentale, misure ed analisi dati}


\begin{wrapfigure}{l}{0.38\textwidth}
\begin{center}
\includegraphics[width=0.38\textwidth]{immagini/DSC_0018.JPG}
\caption{Segnale in uscita dal PMT}
\label{oscillo}
\end{center}
\end{wrapfigure}

Come prima operazione si verifica qualitativamente il funzionamento della strumentazione, ponendo la sorgente di $\ce{^{137}_{}Cs}$ sul portacampioni di fronte allo scintillatore, alimentando il PMT a una tensione tipica e visualizzando il segnale in uscita dal PMT all'oscilloscopio. Le scale di tempi e tensione sono settate in accordo ai valori ottenuti nel Par.(\ref{PMT}).
Il segnale visualizzato è riportato in Fig.(\ref{oscillo}). La forma del segnale ha il tipico andamento esponenziale che ci si aspettava.
Si notano contemporaneamente diversi segnali, con ampiezza diversa. Il più alto sarà quindi dovuto al fotopicco mentre i più bassi sono da ricondurre all'effetto Compton.
\subsection{Operazioni preliminari} \label{HV}
 \underline{ 1) HV PMT:} prima di partire con le misure, va scelta la migliore tensione di alimentazione HV del fototubo. Se è troppo bassa il PMT non riesce a formare il segnale, se è troppo alta si aumenta troppo il rumore (e si può danneggiare lo strumento). Ci si aspetta che i conteggi in funzione di HV seguano una S-curve, la tensione ideale sarà quindi all'inizio della parte piatta alta della curva, detta \textit{ginocchio}. Si porta l'uscita del PMT all'amplificatore su cui si setta un guadagno di qualche Volt. L'uscita amplificata è quindi portata al contatore, settato su una threshold di $0.4V$ e su tempo di $2'$. Si contano i segnali al variare di HV, si ottiene il grafico in Fig.(\ref{counts}). Viene anche eseguito un Fit con una S-curve (Error Function), tuttavia è da rigettare. Si sceglie "ad occhio" la tensione \textit{HV=1003V} come posizione del ginocchio e quindi come tensione operativa dell'esperimento.
 
 \begin{figure}[hbtp]
 \centering
 \includegraphics[scale=0.75]{immagini/HV.pdf}
 \caption{Curva per individuare la tensione operativa}
 \label{counts}
 \end{figure}
 
 \underline{ 2) Amplificazione:} una volta scelta la tensione di alimentazione, si deve settare un opportuno guadagno dell'amplificatore. Guardando la Tab.(\ref{nuclidi}) si vede che il $\ce{^{60}_{}Co}$ è l'elemento che produce gamma più energetici. Pertanto il segnale di fotopicco del $\ce{^{60}_{}Co}$ dovrà avere la tensione più alta accettata dall'MCA. Quest'ultimo accetta un valore massimo di $10V$ quindi servendosi dell'oscilloscopio si setta un'amplificazione che produca un fotopicco del $\ce{^{60}_{}Co}$ di $9V$ \footnote{Non si è preso nota dei valori esatti di amplificazione di coarse e fine ma corrisponderanno ad un fattore di Gain di circa 3000}.

\subsection{Taratura MCA}
L'ultimo operazione prima dell'acquisizione degli spettri è la scelta del numero $n$ di bit di risoluzione dell'MCA. Se questo è grande lo spettro sarà più dettagliato e l'errore sull'energia più piccolo ma verrà richiesto maggior tempo di acquisizione; comportamento speculare per $n$ piccolo. Con alcuni brevi test osservando lo spettro del Cesio si decide che il miglior compromesso è $n=9$ quindi $512$ canali per un tempo di acquisizione di $5'$. Inoltre si decide di eliminare i conteggi presenti nel canali $<48$ (cfr. App.(\ref{a})).

Gli spettri ottenuti sono riportati e commentati in App(\ref{a}).
Sono qui riportati per brevità nel grafico complessivo di Fig.(\ref{tot}): data la differenza di attività fra le sorgenti impiegate è necessario un grafico logaritmico (questo è anche più adatto per la determinazione della spalla Compton come descritto nel Par.(\ref{rel}).

\begin{figure}[hbtp]
\centering
\includegraphics[width=.6\textwidth , angle=-90]{immagini/LogPlot_FLAT.pdf}
\caption{Grafico complessivo degli spettri delle sorgenti e relativi valori tabulati delle energia di fotopicchi e spalla Compton. I valori sottolineati sono i fotopicchi impiegati nella retta di taratura. I valori cerchiati sono i valori di spalla Compton impiegati nel Par.(\ref{rel}). Non sono stati acquisiti i conteggi per $ch<48 \approx 100 keV$.}
\label{tot}
\end{figure}

\begin{figure}[htbp]
\hspace{-10mm}%
\begin{minipage}[c]{.45\textwidth}
%\centering\setlength{\captionmargin}{0pt}%
\centering
\begin{tabular}{|c|c|c|c|}
\hline
                    & Energia {[}MeV{]} & Canale{[}\#{]} & FWMH {[}\#{]}\\ \hline
Ba                  & 0.356             &      127$\pm$1 & 9\\ \hline
\multirow{2}{*}{Na} & 0.511             &     179$\pm$1 & 14\\ \cline{2-4}
                    & 1.275             &     441$\pm$1 & 21\\ \hline
Ce                  & 0.662             &   229$\pm$1  & 16 \\ \hline
\multirow{2}{*}{Co} & 1.173             &   395$\pm$1   & 20\\ \cline{2-4} 
                    & 1.332             &     450$\pm$1  & 23 \\ \hline
\end{tabular}
\caption{Misure ricavate usando la funzione "ROI" del software }
\label{tarval}
\end{minipage}%
\hspace{13mm}%
\begin{minipage}[c]{0.45\textwidth}
%\centering\setlength{\captionmargin}{0pt}%
\includegraphics[scale=0.85]{immagini/taratura.pdf}
\caption{Retta di taratura dell'MCA $ \mathbf{@} \; HV=1003V\; e \; Gain=\ce{^{60}_{}Co} \leftrightarrow 9V$. Assunto errore a posteriori.} 
\label{tarret}
\end{minipage}
\end{figure}

Per calibrare l'MCA si deve far corrispondere ad ogni valore di energia del fotopicco tabulato in Tab.(\ref{nuclidi}), il valore di canale. Quest'ultimo sarà chiaramente il centroide del picco (asse di simmetria della Gaussiana o Lorentziana che approssima il picco) ed è ricavato usando la funzione \textit{ROI} integrata nel software. I valori ottenuti sono riportati in Tab.(\ref{tarval}) e la relativa retta di taratura ottenuta con Fit lineare è mostrata in Fig(\ref{tarret}). Per il $\ce{^{133}_{}Ba}$ si è considerato il solo picco più distinto, in quanto per gli altri non avevano la tipica forma piccata, causa sovrapposizione con altro segnale e quindi il valore ottenuto con la \textit{ROI} non sarebbe stato affidabile (inoltre sono valori vicini fra loro quindi non particolarmente utili al fine di un calcolo di Fit).





Si ottiene la retta di taratura:

\begin{gather} \label{eqtar}
\boxed{ \mathbf{E} [MeV]=\mathbf{-0.022} (\pm 0.013) [MeV]+\mathbf{0.00297} (\pm 0.00004) \left[\frac{MeV}{ch}\right] \cdot \mathbf{ch}} \\ \label{eqtarpar}
\mathbf{@} \; HV = 1003V\; , \; Gain=\ce{^{60}_{}Co} \leftrightarrow 9V , \; T=290 °K\\ \label{eqtar2}
\sigma_E (ch)= \sqrt{\sigma_a^2 + (ch \cdot \sigma_b )^2 + (b\cdot \sigma_{ch})^2 + 2 \cdot b \cdot ch \cdot C_{a,b}} \approx \sqrt{\sigma_a^2 + (ch \cdot \sigma_b )^2 + (b\cdot \sigma_{ch})^2} \\\label{eqtar3}
\text{per un fotopicco} \rightarrow  \sigma_{ch} = 1 \rightarrow \sigma_E \approx 20keV \rightarrow  \frac{\sigma_E}{E} \approx 2 \div 4 \%
\end{gather}

Nella Eq.(\ref{eqtar}) l'intercetta negativa è dovuta al \textit{piedistallo} dell'ADC. \newline
I parametri operativi riportati in Eq.(\ref{eqtarpar}) influenzano la retta di taratura. In generale un sistema di spettroscopia andrebbe studiato con sistemi di feedback che rendano la retta di taratura indipendente da questi paramentri (in range di variazione ragionevoli). \newline
Nella Eq.(\ref{eqtar2}) la covarianza ottenuta ha ordine di grandezza $10^{-7}$ e quindi può essere trascurata.
Il software di acquisizione fornisce valori dei picchi con incertezza sempre $<1ch$, questo motiva l'approssimazione della Eq.(\ref{eqtar3}): l'incertezza sull'energia è variabile con il valore del canale e l' Eq.(\ref{eqtar3}) vuole solo sottolineare qual è l'ordine di grandezza dell'errore della strumentazione nella determinazione del singolo picco. 

Si può testare la bontà della retta di calibrazione, confrontando il valore che si ottiene ad esempio per una spalla Compton (cfr. metodo Par.(\ref{rel})) e confrontarlo con il valore di riferimento di Tab (\ref{nuclidi}). Per il $\ce{^{137}_{}Cs}$ si ha un valore di $ch=155 \pm 6$. Applicando Eq.(\ref{eqtar}) ed Eq.(\ref{eqtar2} si ottiene $E=0.48 \pm 0.02 MeV$ compatibile con quello atteso $E=477 MeV$.

Si può quindi procedere ad impiegare la calibrazione per la determinazione dei picchi di una sorgente ignota.

\pagebreak

\subsection{Analisi di una sorgente ignota}

\begin{figure}[hbtp]
\centering
\includegraphics[scale=0.6,angle=-90]{immagini/ignota_FLAT.pdf}
\caption{Spettro della sorgente ignota. I valori cerchiati corrispondono ai picchi tipici di \ce{^{152}_{}Eu}}
\label{ignota}
\end{figure}


Una volta trovato il canale associato al picco (ad esempio con la funzione \textit{Find Peak} di \textit{Mathematica}) si applica la formula \ref{eqtar} per trovare il corrispondente valore di energia. Si riportano i valori per comodità in Fig.(\ref{ignota}) dove è omessa l'incertezza (circa costante a $0.02 \; MeV$).

Per trovare l'isotopo in questione si può usare il comodo software \textit{Radiation Search} del sito "The Lund/LBNL Nuclear Data Search" reperibile al seguente indirizzo : 

http://nucleardata.nuclear.lu.se/toi/radSearch.asp. 

E' ovvio considerare come valore iniziale di ricerca il picco più alto, stimato a $0.11 MeV$.

Impostando quindi come parametri di ricerca $E_{\gamma}= 120 \pm 20 \; keV$, e un tempo di decadimento ragionevole per sorgenti "didattiche" $t_{\frac{1}{2}}=10 \div 100 \; y$ si vede nella lista di possibili candidati che l'elemento con probabilità\footnote{La probabilità di decadimento sarà proporzionale all'intensità del picco e quindi all'attività riferita a quel picco. Si tratta solitamente di decadimento in cascata quindi non concorrenti, pertanto $\Sigma prob \neq 1$ } di decadimento decisamente maggiore è \textbf{\ce{^{152}_{}Eu}} che ha un picco gamma di $E_{\gamma}=121 keV$ con probabilità $28 \%$.
Si può quindi accedere al database di \ce{^{152}_{}Eu} e in effetti si verifica che i primi 5 valori più intensi di picchi corrispondono a quelli dello spettro misurato e hanno altezza proporzionale all'intensità indicata (valori cerchiati in Fig.(\ref{ignota})).

Per quanto riguarda i rimanenti picchi i valori di $E_{\gamma}=70 keV$ e $E_{\gamma}=80 keV$ possono essere ricondotti sempre usando \textit{Radiation Search} a una fonte di \ce{^{44}_{}Ti} o a una somma di questi con il tipico $K_{\alpha}=75keV$ del Pb.
Il picco a $E_{\gamma}=30 keV$ dovrebbe essere dovuto al backscattering del fotone di energia $E_{\gamma}=110 keV$, infatti applicando la \ref{back} si ottiene $E_{110} ^{BS} \approx 33 keV$.
I due picchi rimanenti sono picchi-somma.








\subsection{Misura della risoluzione energetica del rivelatore}

Con la funzione \textit{ROI} del software di acquisizione si ricavano i centroidi dei fotopicchi delle sorgenti e le corrispondenti FWMH (Tab.(Fig.(\ref{tarval}))).

 Si vuole verificare se la risoluzione varia con l'energia con un andamento $\propto \frac{1}{\sqrt{E}} $. Il Fit mostrato in Fig.(\ref{risfig}) conferma questo andamento. Si può stimare quindi che la risoluzione del rivelatore NaI(Tl) si attesta a $6 \div 7 \% $ che è appunto un valore tipico per questo tipi di materiali.

\begin{figure}[hbtp]
\centering
\includegraphics[scale=0.9]{immagini/risoluzione.pdf}
\caption{Andamento della risoluzione del rivelatore}
\label{risfig}
\end{figure}

\pagebreak

\subsection{Misura della massa dell'elettrone e osservazione della Relatività Speciale} \label{rel}

Per determinare il valore di canale della spalla Compton e relativo errore non si può chiaramente usare la funzione \textit{ROI}, si ricorre quindi ad un metodo "a mano": si misura sul grafico il punto $ch1$ di inizio della discesa della spalla e il punto $ch2$ di metà altezza della discesa.
Il valore $C$ della  spalla Compton sarà il punto medio di $ch1$ e $ch2$ e l'errore associato sarà la semiampiezza $\frac{ch2-ch1}{2}$.
Ricordando che la l'energia cinetica dell'elettrone si misura proprio tramite l'energia da esso rilasciata nello scintillatore, si procede a calcolare $T_e$ con relativi errori usando le Eq.(\ref{eqtar}).
Si calcolano i corrispondenti valori di $m_e c^2$ usando la Eq.(\ref{massa}) e relativi errori con la formula di propagazione errori. 
I risultati sono riassunti in Tab.(\ref{masstab}).

\begin{table}[]
\centering
\renewcommand\arraystretch{1.2}
\begin{tabular}{|c|c|c|c|c|c|c|}
\hline
 & $E_{\gamma} [MeV]$ & $ch1 [\#]$ & $ch2 [\#]$ & $C_{Comp} \pm \sigma_C [\#]$ & $T_e \pm \sigma_{T_e} [MeV]$ & $m_e c^2 \pm \sigma_{mc^2} [MeV]$   \\ \hline
 $\ce{^{22}_{} \mathbf{Na}} (I)$& 0.511 &114  &130  & $122 \pm 8 $ & $ 0.34\pm 0.03$ &   $0.67 \pm 0.11$   \\ \hline
 $\ce{^{137}_{} \mathbf{Cs}}$ & 0.662 & 158 & 171 & $164 \pm 6$  & $ 0.47\pm 0.02$ &  $0.78 \pm 0.08 $    \\ \hline
  $\ce{^{60}_{} \mathbf{Co}}$& 1.173 & 311 & 339 & $325 \pm 14 $ & $ 0.95\pm 0.04$ &  $1.02 \pm 0.12 $    \\ \hline
  $\ce{^{22}_{} \mathbf{Na}} (II)$& 1.275 & 349 & 376 & $362 \pm 13 $ & $ 1.06\pm 0.04 $ &   $ 1.05 \pm 0.11$   \\ \hline
\end{tabular}
\caption{Valori per la misura della massa dell'elettrone}
\label{masstab}
\end{table}

Con i valori di Tab.(\ref{masstab}) si ottiene il Fit riportato in Fig.(\ref{massfit}). Il risultato ottenuto è:

\begin{equation}
\boxed{\mathbf{m_e c^2 }= (0.52 \pm 0.12) + (0.51 \pm 0.17) \cdot \mathbf{T_e}  \quad [MeV] }
\end{equation}

I valori ottenuti sono compatibili con quelli attesi. Tuttavia, la misura della massa a riposo dell'elettrone ha un errore del $20 \%$ che è piuttosto alto. Questo è dovuto all'indeterminazione sul canale della Spalla Compton che si propaga nel calcolo di $T_e$ e ancora si propaga nel calcolo di $m_e c^2$.

\begin{figure}[hbtp]
\centering
\includegraphics[scale=0.8]{immagini/massa_elett.pdf}
\caption{Fit lineare per la misura della della massa dell'elettrone}
\label{massfit}
\end{figure}


\clearpage
\appendix 
\section{Spettri acquisiti} \label{a}
 
 \begin{figure}[h]
 \centering
 \subfigure[Cesio-137]
   {\includegraphics[scale=0.93]{immagini/Cs137.pdf}}
 \subfigure[Cobalto-60]
   {\includegraphics[scale=0.93]{immagini/Co60.pdf}}
  \subfigure[Bario-133]
   {\includegraphics[scale=0.93]{immagini/Ba133.pdf}}
  \subfigure[Sodio-22]
   {\includegraphics[scale=0.93]{immagini/Na22.pdf}}
 \end{figure}
 
Il software di acquisizione fornisce la lista di coppie $conteggi-canale$. I grafici qui riportati sono generati con \textit{Mathematica} usando la funzione \textit{Joined Point} per motivi estetici.Non è stato sottratto il rumore poichè si è supposto non influenzare la determinazione della posizione dei fotopicchi.

I grafici hanno asse delle ordinate scalato con il valore massimo di fotopicco, dato che hanno attività molto diverse fra loro. I valori di conteggi e relativi rate non sono necessari ai fini della presente esperienza. Allo stesso modo non sono stati presi valori con $ch<48$ in quanto questi corrispondono a valori circa $ < 100 keV$, zona in cui ogni segnale è "sommerso" nel picco dovuto dai fotoni $K_{\alpha} \approx 75 keV$ di fluoerescenza dello schermo di Piombo, dovuto all'effetto fotoelettrico compiuto dai raggi-$\gamma$. Ad ogni picco fotoelettrico corrisponde una spalla Compton eccetto quando essa si trova sommmersa sotto altro segnale come nel Cobalto-60 e nel Bario-133.
Infine, negli spettri di Cobalto-60 e Cesio-137 è particolarmente distinto alle basse energie il picco di back-scattering dovuto al raggio-$\gamma$ che viene riflesso di $180°$ per scattering Compton con la schermatura di Piombo. Il valore del fotone di back-scattering si ricava dalla Eq.(\ref{back}).
\end{document}
