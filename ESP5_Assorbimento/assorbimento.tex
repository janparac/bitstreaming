\documentclass[12pt,a4paper,openright,twoside]{article}

\usepackage[italian]{babel}
%\usepackage[latin1]{inputenc}
\usepackage[utf8x]{inputenc}
%\usepackage[applemac]{inputenc}

\usepackage[]{hyperref} %collegamenti ipertestuali + url internet
\usepackage{url}
\usepackage{xcolor} 
\hypersetup{ %personalizzazione collegamenti ipertestuali
    colorlinks = false,
    linkbordercolor = {yellow},
}

\usepackage[centertags]{amsmath}
\numberwithin{equation}{section} %numerare le formula fino alla sezione (scrivere SUBSECTION per arrivare alla sottosezione)
\usepackage{amsfonts}
\usepackage{amssymb}
\usepackage{amsthm}
\usepackage{amstext}
\usepackage{enumitem}%per gli elenchi numerati romani
\usepackage{multirow}%per tabelle con righe unite
\usepackage{colortbl}
\usepackage[version=2]{mhchem} %nuclidi
\usepackage{footnote} %per fare note in tabella
\makesavenoteenv{table} %per fare note in tabella
\usepackage[T1]{fontenc} % mette le due virgolette uguali e dritte (forse)

\usepackage[pdftex]{graphicx}%per le immagini

\DeclareGraphicsExtensions{.pdf,.png,.jpg,.mps}
\usepackage{fancyhdr}
\usepackage{indentfirst}
\usepackage{newlfont}
\usepackage[small,bf]{caption}
\usepackage{lscape}
\usepackage{multicol}
\usepackage{units}%per il segno di frazione figo

\usepackage{subfigure}%matrici di immagini
\usepackage{wrapfig}%per le immaggine nel a lato testo
\usepackage[percent]{overpic}%per mettere il testo davanti alle figure

%PAGE formatting
\setlength{\hoffset}{-25,4mm}
\setlength{\captionmargin}{10pt}
\evensidemargin=17mm
\oddsidemargin=17mm
\linespread{1}
\textwidth=175mm
\textheight=230mm

\hyphenation{}


\pagestyle{fancy}
%\renewcommand{\chaptermark}[1]{\markboth{\thechapter.\ #1}{}}
%\renewcommand{\sectionmark}[1]{\markright{\thesection \ #1}{}}
\fancyhf{}
\fancypagestyle{plain}{\fancyhead{}}
\fancyhead[LE, RO] {\bfseries\thepage}
\fancyhead[LO]{\bfseries\rightmark}
\fancyhead[RE]{\bfseries\leftmark}
\renewcommand{\headrulewidth}{0.5pt}
\renewcommand{\footrulewidth}{0pt}
\addto\captionsitalian{\renewcommand{\chaptername}{}}

\title{\textbf{Misura del coefficiente di assorbimento della radiazione-$\gamma$ in Pb e Al}}
\date{\small{ Turno di misura\\
	23/2/2017}}
\author{Gruppo 3: \\  
		Stefano Paracchino 
		}



\begin{document}
\maketitle
    

%--------------------INDICE---------------------------------------------------------------------
\tableofcontents                        %crea l'indice
%%%%%%%%%%%%%%%%%%%%%%%%%%%%%%%%%%%%%%%%%imposta l'intestazione di pagina

%%%%%%%%%%%%%%%%%%%%%%%%%%%%%%%%%%%%%%%%%non numera l'ultima pagina sinistra
\pagenumbering{arabic}
%--------------------------------------------------------------------------------------------------------
\pagebreak
\section{Scopo dell'esperienza}
L'obiettivo è verificare la legge fondamentale di attenuazione dell'intensità $I$ della radiazione nell'attraversare materiali si spessore $x$ e quindi misurare il \textit{coefficiente di assorbimento} $\mathbf{ \mu} \; [cm^{-1}]$

\begin{equation} \label{ass}
I(x)=I_0 e^{- \mu x}
\end{equation}

con $I_0$ intensità della radiazione incidente sul materiale.
L'intensità è direttamente proporzionale al numero di fotoni, i quali sono le particelle che verranno rilevate dalla strumentazione di questa esperienza.

I valori di $\mu $ ottenuti verranno confrontati con il database del NIST\footnote{Per Pb: http://physics.nist.gov/PhysRefData/XrayMassCoef/ElemTab/z82.html

Per Al:  http://physics.nist.gov/PhysRefData/XrayMassCoef/ElemTab/z13.html}, che fornisce il \textit{coefficiente di attenuazione di massa} $\mu_a =\frac{\mu}{\rho} \; [\frac{cm^2}{g}] $ dove $\rho \; [\frac{g}{cm^3}] $ è la densità del materiale \footnote{Per Pb e Al: http://ishtar.df.unibo.it/mflu/tafel/densit.html}.
I valori di $\mu_a$ sono tabulati per vari valori di energia del fotone. In questa esperienza si utilizzeranno fotoni di energia $E_{\gamma}=0.662 MeV$ quindi interpolando i valori forniti per $E_{\gamma}=0.6 MeV$ e $E_{\gamma}=0.8 MeV$ si ottiene:

\begin{gather}
\mathbf{Al} \qquad \rightarrow \qquad \mu = 0.07504 \; [\frac{cm^2}{g}] \cdot 2.70 \; [\frac{g}{cm^3}] = \boxed{ 0.2026 \; [cm^{-1}]} \\
\mathbf{Pb} \qquad \rightarrow \qquad \mu = 0.1136 \; [\frac{cm^2}{g}] \cdot 11.34 \; [\frac{g}{cm^3}] = \boxed{ 1.288 \; [cm^{-1}]}
\label{mu}
\end{gather}

I valori sono assunti senza errore.


\section{Strumentazione e Campioni}

\begin{enumerate}
\item \underline{sorgente}: i raggi-$\gamma$ sono prodotti dall'isotopo radioattivo $\ce{^{137}_{55}\mathbf{Cs}}$ poichè ha un unico fotopicco ben definito di energia $E_{\gamma}=0.662 MeV$
\item \underline{Campioni assorbitori}: $7$ placchette di Pb, $4$ placchette di Al. Hanno area rettangolare (lato di qualche cm ) e posso essere posizionate nell'apposito portacampioni fra sorgente e rivelatore.
Possono essere messi in successione a formare differenti spessori totali, ovvero la variabile $x$ di Eq.(\ref{ass}).
Gli spessori $x$ delle singole placchette sono riportati in Tab.(\ref{spess}).
\item \underline{rivelatore}: Rivelatore a scintillazione $NaI(Tl)$ e tubo $PMT$ (per dettagli vedi \textit{Esperienza Taratura MCA}). Considerando il principio di funzionamento di un rivelatore di questo tipo, si osserverà in questa esperienza una variazione del profilo energetico al variare dello spessore $x$. In particolare si alzerà la "valle" fra spalla Compton e fotopicco a causa appunto dell'aumento di probabilità del primo effetto.

\item \underline{MCA}: per la descrizione generale di un MCA ci si riferisce a \textit{Esperienza Taratura MCA}, in questo caso, il MultiChannel Analyzer è montato su un modulo del rack; non può trasferire i dati su memoria esterna per successivo processamento quindi gli spettri acquisiti sono analizzati solo nel tempo di svolgimento dell'esperienza. Per far ciò si può far scorrere lungo i canali un cursore verticale che fornisce il corrispondente valori conteggio. Analogamente fornisce i conteggi "di area" presenti nell'intervallo di canali fra due cursori.

Può contenere nella sua memoria interna 8192 canali quindi una misura con risoluzione a 13 bit. Se si adottano risoluzioni inferiori, la memoria può essere suddivisa in sottosezioni senza essere sovrascritta (e.g. 16 spettri da 512 canali). Questa funzione è molto utile per poter analizzare uno spettro mentre se ne acquisisce un altro e quindi velocizzare la presa dati.

Quando si imposta il tempo di acquisizione è anche possibile definire se vada considerato come \textit{tempo reale} oppure come \textit{tempo vivo} ( cioè tempo che non considera gli intervalli di \textit{tempo morto}, e.g. il tempo di conversione dell'ADC e il tempo di scrittura in memoria).
Accetta segnali fino a $8V$.

\item \underline{Alimentatore:} modulo NHQ-203M fornisce che l'alta tensione al PMT.
\item \underline{Amplificatore}: modulo amplificatore ORTEC con $Z=90 \Omega $ (per dettagli vedi \textit{Esperienza Taratura MCA}).
\item \underline{Oscilloscopio:} Oscilloscopio analogico Tektronix $BP 100 MHz$ utilizzato per il controllo del segnale in uscita dal MCA.
\end{enumerate}

\section{Acquisizione misure ed analisi dati}
\subsection{Operazioni preliminari}
Si misurano gli spessori dei campioni di materiale assorbitore con calibro cinquantesimale. Le misure sono riportate in Tab.(\eqref{spess}). Data l'irregolarità della forma delle placchette si stima un inceretezza sulla misura di $0.05mm$ per Al e $0.1mm$ per Pb.

\begin{table}[]
\centering

\begin{tabular}{|c|c|c|c|c|c|c|c|}
\hline
\multicolumn{8}{|c|}{Allumino}                                         \\ \hline
Sigla             & $A_a$     &$ B_a$     & $C_a $    & $D_a$     &      &      &      \\ \hline
$x\pm 0.005 \;[cm]$ & 1.002 & 1.002 & 0.884 & 1.002 &      &      &      \\ \hline
\multicolumn{8}{|c|}{Piombo}                                           \\ \hline
Sigla             &$ A_p$     & $B_p $    & $C_p$     & $D_p $    & $E_p   $ & $F_p  $  & $G_p $   \\ \hline
$x\pm 0.01 \; [cm]$  & 0.20  & 0.21  & 0.21  & 0.22  & 0.21 & 0.28 & 0.21 \\ \hline

\end{tabular}
\caption{ spessori $x$ dei campioni assorbitori}
\label{spess}
\end{table}

Il PMT viene alimentato alla tensione consigliata di $\mathbf{HV=1000V}$.
Si posiziona la sorgente e si guarda il segnale in uscita dal PMT all'oscilloscopio per decidere il valore di trigger da impostare nell'amplificatore.
Si amplifica il segnale di un valore opportuno compatibile con il fondoscala di $8V$.
Si imposta l'MCA in modo che divida la memoria in 16 slots da \textbf{512 }canali (quest'ultima è una soluzione adatta a soddisfare il compromesso fra tempo di acquisizione e "costruzione" di un picco ben definito). Si sceglie un tempo di acquisizione vivo di {$\mathbf{t_a=5min}$}.

\pagebreak
\subsection{Acquisizione conteggi}

Si acquisiscono gli spettri energetici variando lo spessore di materiale attraversato. Si misurano i conteggi di picco, posizionando un cursore sul picco e i conteggi di area posizionando due cursori nei punti di FWMH.
Le misure sono riassunte in Tab.(\ref{count}).


\begin{table}[]
\centering

\begin{tabular}{|c|c|c|c|}
\hline
\multicolumn{4}{|c|}{Senza campioni}                                                                 \\ \hline
                                     & x {[}cm{]} & conteggi picco {[}\#{]} & conteggi area {[}\#{]} \\ \hline
                                     & 0          & 1091                    & 38384                  \\ \hline
\multicolumn{4}{|c|}{\textbf{Allumino}}                                                                       \\ \hline
Campioni                             & x {[}cm{]} & conteggi picco {[}\#{]} & conteggi area {[}\#{]} \\ \hline
$A_a$                                &   $1.002 \pm 0.005$         & 885                     & 33261                  \\ \hline
$A_a+B_a$                            &   $2.004 \pm 0.007$          & 731                     & 27070                  \\ \hline
$A_a+B_a+C_a$                        &   $2.888 \pm 0.009$          & 663                     & 22142                  \\ \hline
$A_a+B_a+C_a+D_a$                    &   $3.89 \pm 0.01$          & 558                     & 20377                  \\ \hline
\multicolumn{4}{|c|}{\textbf{Piombo}}                                                                         \\ \hline
Campioni                             & x {[}cm{]} & conteggi picco {[}\#{]} & conteggi area {[}\#{]} \\ \hline
$A_p$                                &     $0.20 \pm 0.01$       & 873                     & 32001                  \\ \hline
$A_p+B_p$                            &     $0.41 \pm 0.01$        & 734                     & 24973                  \\ \hline
$A_p+B_p+C_p$                        &     $0.62 \pm 0.02$       & 598                     & 20606                  \\ \hline
$A_p+B_p+C_p+D_p$                    &     $0.84 \pm 0.02$        & 450                     & 16079                  \\ \hline
$A_p+B_p+C_p+D_p+E_p$                &     $1.05 \pm 0.02$        & 393                     & 13151                  \\ \hline
$A_p+B_p+C_p+D_p+E_p+F_p$            &     $1.33 \pm 0.02$        & 286                     & 10567                  \\ \hline
$A_p+B_p+C_p+D_p+E_p+F_p+G_p$        &     $1.54 \pm 0.03$        & 234                     & 8739                   \\ \hline
\end{tabular}
\caption{Misure ricavate dall'MCA. L'errore sui conteggi è Poissoniano (non indicato). L'errore dei singoli spessori è propagato sulla somma}
\label{count}
\end{table}



Il calcolo dei Fit esponenziali è riportato in Fig.(\ref{fit}).


\begin{figure}[h]
 \centering
 \subfigure[Alluminio (conteggi picco)]
   {\includegraphics[scale=0.66]{immagini/Al_picco.pdf}}
 \subfigure[Alluminio (conteggi area)]
   {\includegraphics[scale=0.66]{immagini/Al_Area.pdf}}
  \subfigure[Piombo (conteggi picco)]
   {\includegraphics[scale=0.66]{immagini/Pb_picco.pdf}}
  \subfigure[Piombo (conteggi somma)]
   {\includegraphics[scale=0.66]{immagini/Pb_area.pdf}}
   \caption{Fit esponenziali con stima dei parametri e valore di $\chi ^2$}
   \label{fit}
 \end{figure}

I calcoli di Fit mostrano comportamenti analoghi per Pb e Al: con i conteggi di picco il valore di $\chi ^2$ fa concludere che l'ipotesi di andamento esponenziale non ha motivo di essere rigettata con un livello di significatività del $5 \%$.
I valori ottenuti e lo Z-test con il valore atteso sono:

\begin{gather}
\mathbf{Al} \qquad \rightarrow \qquad \mu = 0.171 \pm 0.012 \; [cm^{-1}] \qquad \rightarrow \qquad Z=\frac{|0.171-0.2026|}{0.012}=2.6 \\ 
\mathbf{Pb} \qquad \rightarrow \qquad \mu = 1.00 \pm 0.03 \; [cm^{-1}] \qquad \rightarrow \qquad Z=\frac{|1.00-1.288|}{0.03}=9.6
\label{ris}
\end{gather}

Il coefficiente di attenuazione ottenuto differisce da quello teorico di un $15 \% $ per l'Alluminio e del $20 \% $ per il Piombo.
A rigore il test-Z (effettuato con errore nullo sul vaore atteso) ci dice che i due valori non sono compatibili (l.s. $5\%$).
E' importante osservare che anche aumentando ulteriormente l'errore sull'ascissa $x$ il risultato per $\mu$ non cambia significativamente.
Analogamente anche si assume l'errore a posteriori $\sigma_{post}= \sqrt{\chi_{rid}} \cdot \sigma$ non si ottiene un risultato migliore.

Nei Fit che usano i conteggi di area i risultati per $\mu$ variano di poco e in questo caso l'ipotesi di andamento esponenziale va sicuramente rigettata (valore di $\chi ^2$ alto a causa dell'errore molto piccolo $ \sigma_{ poisson} ^{rel} \approx \frac{1}{\sqrt{n}}$).


\end{document}